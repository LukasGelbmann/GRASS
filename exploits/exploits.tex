% Exploits report

\documentclass{article}

\usepackage[utf8]{inputenc}

\title{Exploits}
\author{Simon Le Bail-Collet, Lukas Gelbmann, Björn Gudmundsson}
\date{May 2019}

\begin{document}

\maketitle


Our first three exploits hijack the control flow, while the last two open a calculator.

Exploit 3 (the format string exploit) only works with ASLR disabled.
The other exploits with even with ASLR disabled because we compile with the \texttt{-static} flag.
This flag doesn't make exploiting our code harder (if anything, it becomes easier).


\section{Buffer overflow (ping)}

The \texttt{ping} command uses \texttt{sprintf()} on a small buffer when the host is not known.
The host name, which is user-controlled, is used as a parameter to \texttt{sprintf()}.
Since no length checks are performed on the string, the \texttt{sprintf()} can overflow the buffer.

We provide a very long host name to overflow the buffer and overwrite the return instruction pointer to hijack the control flow.


\section{Buffer overflow (``Command not found'')}

When formatting the exception description for the case where an incoming command is not specified by the GRASS protocol, no length checks are performed.
\texttt{sprintf()} is used in a similar manner as in ping and an easy buffer overflow exploit can be performed.

We provide a very long invalid command to again overwrite the return instruction pointer.


\section{Format string (ls)}

We were able to exploit our format string bug a day after handing in our code.
Inexperienced developers as we are, we ``accidentally'' used \texttt{sprintf()} with a user-controlled format string.
This allow us to write to an arbitrary address.
With ASLR disabled, we can reliably overwrite the return instruction pointer.


\section{Command injection (grep)}

This exploit takes advantage of how our implementation of the GRASS protocol uses the \texttt{ls} system command to traverse all child directories of the current directory.
The \texttt{grep} command does not surround its argument with single-quotes, allowing for arbitrary command execution.

The exploit goes as follows: After a successful login, the user issues a \texttt{mkdir} command
and names the directory something with a semicolon, followed by a shell command. Then the user issues a \texttt{grep} command in the parent directory, which will run \texttt{ls} with a user-provided argument et voilà.


\section{Free choice: command injection (rm)}

This exploit uses a vulnerability in how directories are made and deleted in our GRASS protocol implementation.

In the \texttt{mkdir} command, we made an implementation decision to remove any single-quote characters in the argument and surround the argument with single-quotes in the name of security.

The exploit takes place in \texttt{rm}.
It takes advantage of the fact that as carefree developers we use the argument to \texttt{rm} in a system call, surrounded by single-quotes but not with single-quotes removed from the argument string.

Before we perform this unsafe call, though, we have a check that the directory exists.
In this check, the single-quotes are removed.

The exploit thus goes as follows: After a successful login, the user creates a directory that has a semi-colon and then a command to open a calculator.
Then the user issues a corresponding \texttt{rm} command that has a closing quote character and allows for arbitrary command execution since the check that the directory exists succeeds.

\end{document}
